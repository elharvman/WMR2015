\section{Strategy}
The strategy of the 2015/16 Warwick Mobile Robot team was based around the following points;
\begin{itemize}
\item{Reviewing Orion, the robot from the 2014/15 WMR team and deciding which aspects require redesign and improvement.}
\item{Base our designs for Cyclone on Orion as much as possible so we are able to reuse some of Orion’s components.}
\item{Building bridges between suppliers and others by means of meetings and outreach events to raise awareness of WMR projects and create relationships between relative suppliers for this year's project and for any future WMR projects.}
\item{Completing the project with a fully functioning robotic vehicle which can be used as a basis for next year’s team to make final additions in order enter into competition.}
\end{itemize}
\subsection{Aims and Objectives}
Our aim is to design, test and manufacture an urban search and rescue robot for use in disaster zones to locate and aid trapped survivors, and to raise awareness for the capabilities of search and rescue robots in these situations. \par
To achieve this aim, our objectives this year are as follows;
\begin{enumerate}
\item{To further develop the previous year’s robotic vehicle to improve its functionality and search and rescue capabilities.}
\item{To conclude this project with a vehicle which can be used as a platform for next year’s team to finalise the robot for means of a competition entry.}
\item{To improve the awareness of search and rescue robots and to inspire the next generation to enter into the world of robotics.}
\end{enumerate} \par
\subsection{Team Structure}
This year’s team consisted of six fourth year students from various disciplines within the School of Engineering at the University of Warwick. The engineering discipline distribution was; two mechanical, two mechanical \& manufacturing, an electronic and a systems. \par
The robot chassis was split into three modules and generated  individual tasks for the four mechanical engineering team members; the chassis, drivetrain, suspension and dynamic tension systems. The entirety of the electronics work was performed by the sole electronic engineer in the team and the systems engineer was responsible for communications, both electrical and mechanical. Collaboration between team members was essential throughout this project, especially between the mechanical members due to the large level of overlap between their respective tasks. \par
Each team member also had an administrative role in order to keep the team running effectively and efficiently and to avoid the time of one member of the team being taken up by all of these tasks.. These roles included a project manager, a secretary, a finance director, a sponsorship manager, a social media officer and outreach officer. \par
\subsection{Project Lifetime}
In the time the team had to complete the project, it was clear that there would not be sufficient time to produce a robot that would be able to compete in a competition. The team decided they will build a robot up to a stage where it could be finalised and taken to the competition by the 2016-17 team. It was therefore important that the robotic vehicle produced by the end of this year be well designed and adaptable to avoid rework for next year’s team. \par
There was 30 weeks in total to review Orion, design for Cyclone, manufacture, assemble and produce the technical report for this project. The designs for Cyclone were finalised by week 20 and ready for manufacture by week 24, which was later than planned due to complications. The manufacturing was then completed by week 28 and assembled in the same week ready for testing in week 29. The testing, evaluating and report writing was then completed by the end of week 30. \par
\subsection{Management and Logistics}
The team held regular meetings, with all members initially but it was decided that the attendance of the entire team was not always necessary and was sometimes counterproductive and so most later meetings only involved the team members that were required for the particular discussion topic. There were also weekly meetings with all the team members along with the project supervisor in order to update the supervisor on project progress. This meeting was often attended by a technician in order to provide manufacturing expertise. \par
Google Drive acted as the storage facility for all relevant files, such as manufacturing drawings and CAD files. Meeting minutes were recorded by the secretary and uploaded to the Drive area following the meeting for members to review. In order to ensure that contact was always kept between team members, an instant messaging application called Slack was used to update each other of progress. \par
For the technical report, each section was written using Google Docs so that comments and changes could be easily made to the documents by the reviewers. Once all sections were completed, the full report was compiled using LaTex, and after several iterations, the final report was completed.