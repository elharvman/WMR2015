\subsection{Design For Manufacture}
\subsubsection{What is Design for Manufacture?}
Design for manufacture is part of Design for Manufacture and Assembly (DFMA). DFMA has three main aspects \cite{Boothroyd10}:

\begin{enumerate}
\item Form a foundation for concurrent engineering studies, providing direction to designers to simplify products to reduce the cost of manufacture and assembly and quantify improvements.
\item As a benchmarking tool to compare products for ease of both manufacturability and assembly.
\item As a “should-cost” tool for cost control and supplier negotiation.
\end{enumerate}
Firstly, concurrent engineering is a methodology that allows engineering effort to be integrated with the life cycle of a product and means that these efforts are carried out at the same time. This links upstream and downstream design processes and takes all stages of a product’s life-cycle into account from the outset \cite{Kusiak93}.

INSERT CONCURRENT ENDINEERING IMAGE


Design for manufacture takes into account the limitations of manufacturing processes as well as considers which processes, materials and techniques are best suited to produce a component most efficiently. Because of this it is critical that Design for Manufacture is used in the product life cycle as early as possible. In DFMA more time is spent in the early design stages (concept design) to establish the performance criteria and parameters. This reduces the design changes further into the product life cycle and reduces the time taken from product inception to market. DFMA can reduce to the time to market by up to 40\% \cite{Boothrody10}. Additionally, there should be more time spent on design in the early stages of the lifecycle because it is the decisions made here that dictate up to 70\% of the total costs. Intelligent and careful design is the greatest factor in reducing cost \cite{Boothroyd10}.

INSERT DMFA TIME SAVINGS IMAGE

\subsubsection{Concurrent Engineering and Design for Manufacture for Cyclone}
The use of concurrent engineering was appropriate to use because of the modular nature of the robot. The new design, the number modules needed and time constraints meant that design of the robot was split among group members to be carried out concurrently. This meant that during the design of each module, its interactions with the other modules had to be considered from the outset, to minimised changes that would have to be made later. Furthermore, because the WMR urban search and rescue project runs from year to year future changes and adaptations further into the robot’s life-cycle had to be taken into account. For example, a robotic arm should be added in the future meaning that the chassis had to be designed to be stiff and strong enough allow an arm to be mounted to the upper chassis. The suspension was also designed to be capable of functioning with the added weight of an arm, hence its adjustability. The motors were also capable of more torque than strictly needed to account for future weight increases and the circuit boards and control systems capable of running additional sensor and cameras in the future.

Cyclone is almost a brand new robot, very few parts were carried over from Orion of 2015. As a result of this there were many new parts designed and all of these would have to be manufactured. The number of parts that needed to be made meant that there were many opportunities for error and delay, which could set the project back and create problems further into the project, for example at assembly if parts have been made to different specifications. To limit the number of issues that would be encountered throughout the course of the project Design for Manufacture was employed to simplify both the design process, as well as manufacture. This would in turn reduce the overall project costs and also feed into Design for Assembly (Section XX).

There are a number of principles of Design for Manufacture, as described by \cite{Chang05}, some of these were used during the design of Cyclone. How these principles were applied to Cyclone is explained below (TOO CLUNKY):

\begin{enumerate}
\item \textit{Reduce total number of parts.}
\par The majority of parts reduction has come from changing the chassis from Makerbeam to plate aluminium. Orion had over 20 Makerbeam sections to make up just the central chassis section and far more fixings. Cyclone has 5 plates making the central chassis section and XX bolts. The situation is the same for the front and drive sections. The reduction in parts count reduces complexity greatly and made design easier as there were fewer part interactions to consider.
\item \textit{Develop a Modular Design}
\par The robot has a modular design. It is made up of three modules: front, mid and drive. These are connected to each other with 6 bolts. The benefit of the modular design is that the robot can be upgraded over time by swapping modules for new ones, or by removing a module to change its components. For example, the front module could be changed for one with a different set of sensors, for use under different conditions. Furthermore, modularity improves the ease or repair and fault finding, as a module can be quickly removed and all components easily accessed. To allow modularity, all connectors between the modules are standardised and located in one connector plate, allowing them to be quickly connected.
\item \textit{Use Standard Components}
\par The most distinct advantages of standard components are the cost and time benefits. Standard components can be bought off the shelf and do not require extensive engineering effort, allowing focus to be on other areas. There are several example of standard components in Cyclone. The linear bearings used in the dynamic tensioning system are stock parts from Igus bearings, which have simply been machined down to fit the packaging requirements. Another example is the suspension swing-arms; these could have been machined from a billed to a complex design but a using stock channel section meets the requirements but is vastly quicker to manufacture.
\item \textit{Design Parts to be Multifunctional}
\par By designing parts to be multifunctional the parts count and complexity of the robot could be reduced. An example of this is using the chassis itself as a heat-sink. Electronic components, such as the motor controllers, were positioned against large surfaces of the chassis to draw away heat and dissipate it though the chassis to the environment. Another example of this is chassis serving as the mounting points for the motors, rather than having motor mounts fixed to the chassis. The suspension’s primary purpose is to absorb bumps and shocks that the robot is subjected to, to allow it to traverse terrain more quickly and it also protects the robot from damage that uneven terrain could cause.
\item \textit{Design for Ease of Fabrication}
\par All parts have been designed to be as easy to manufacture as possible. As part of this, the majority of parts can be manufactured from one side. This means that fixturing during manufacturing is easy, as the parts to not need to be turned and repositioned, which could cause misalignment. Parts have been designed with stock material size in mind to reduce manufacturing time where possible. Examples of this are the drive shafts, which are 16mm meaning that they do not need to be turned down, only cut to length and additional features added, if needed.
\end{enumerate}
To summarise concurrent engineering and design for manufacture were used in the design process of Cyclone to reduce the time taken to design to robot, as well as the cost and increase simplicity. It was especially important to use these methods because of the number of parts that had to be designed from scratch, as well as the level of complexity required for the robot to have the capabilities needed. 